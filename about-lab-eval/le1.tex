% Created 2024-10-25 Fri 12:54
% Intended LaTeX compiler: pdflatex
\documentclass[11pt]{article}
\usepackage[utf8x]{inputenc}
\usepackage[T1]{fontenc}
\usepackage{graphicx}
\usepackage{longtable}
\usepackage{wrapfig}
\usepackage{rotating}
\usepackage[normalem]{ulem}
\usepackage{amsmath}
\usepackage{amssymb}
\usepackage{capt-of}
\usepackage{hyperref}
\usepackage{minted}
\author{B.V. Raghav}
\date{\textit{[2024-09-11 Wed]}}
\title{Lab Eval 1\\\medskip
\large UCS749: Conversational AI: Speech Processing and Synthesis}
\hypersetup{
 pdfauthor={B.V. Raghav},
 pdftitle={Lab Eval 1},
 pdfkeywords={},
 pdfsubject={},
 pdfcreator={Emacs 29.3 (Org mode 9.6.15)}, 
 pdflang={English}}
\begin{document}

\maketitle
\noindent \textbf{Problem Code}: 202425ODD-UCS749-SESS-LE1-0911 \\[0pt]
\textbf{Problem Title}: Recognise My Voice Commands.

\section{Logistics}
\label{sec:orgd34df8b}
\begin{itemize}
\item Start time: Wed Sep 11 10:00 AM
\item End Time: (Extended) Wed Sep 12 11:59 PM
\item Submission Form: (Updated) \url{https://docs.google.com/forms/d/e/1FAIpQLSf9DzoyzW\_3kQe2FVqRD7RpjbXmk4HnSun\_2LwnWdOggV\_q6g/viewform?usp=pp\_url\&entry.185703634=202425ODD-UCS749-SESS-LE1-0911\&entry.1322657816=Recognise+My+Voice+Commands}
\item Viva Voce: Will be notified later.
\end{itemize}

\section{Task}
\label{sec:org5b1f750}
Consider the paper: \url{https://arxiv.org/abs/1804.03209}

\begin{enumerate}
\item Read and summarise the paper in about 50 words.
\item Download the dataset in the paper, statistically
analyse and describe it, so that it may be useful
for posterity. (Include code snippets in your .ipynb
file to evidence your analysis.)
\item Train a classifier so that you are able to
distinguish the commands in the dataset.
\item Report the performance results using standard
benchmarks.
\item Record about 30 samples of each command in your
voice and create a new dataset (including a new user
id for yourself).  You may use a timer on your
computer to synchronise.
\item Fine tune your classifier to perform on your voice.
\item Report the results.
\end{enumerate}


\section{Deliverables}
\label{sec:org8cf1853}
\begin{enumerate}
\item A PDF Report: (as a part of your Git Repo) named
<ROLL\textsubscript{NO}>-report.pdf
\item Assets: Your pretrained classifier model weights and your
cleaned and well-formed dataset.  This should be a
part of your google drive with read access to your
instructor <bv.raghav@thapar.edu>
\item A demo notebook: (as a part of your Git Repo), that
loads both your model and dataset; and runs to show
the results.
\item The demo notebook should verify the assets using a
checksum (md5/sha/…).  This step verifies that the
assets have not been tampered with at a later stage.
\end{enumerate}


\section{Evaluation}
\label{sec:org9f1dcd4}
\begin{enumerate}
\item Clarity of thought process and presentation.
\item Data processing skills.
\item Model fine tuning/ training skills.
\item Details of progress, as in what were the encountered
problems and how were they solved.
\item How adaptable is your pipeline? (as in, how easy is
it for me to adapt it for my voice)
\item How scalable is your approach? (as in, how easy is
it to scale it to many new voices)
\item Strengths and Shortcomings of your approach.
\end{enumerate}


\section{Note}
\label{sec:orgd18e514}
\begin{enumerate}
\item This is a test of how fast can we report the
performance of a model for a specific task.  The
best performance is not expected; but a holistic
pipeline is.
\item You may improve upon it in future, out of interest;
though it wouldn’t influence your eval.
\end{enumerate}


\section{Struts}
\label{sec:orgb9662c6}
The following tutorials may be a good start point;
there maybe more on the internet.  You are free to
choose.
\begin{enumerate}
\item \url{https://colab.research.google.com/github/pytorch/tutorials/blob/gh-pages/\_downloads/c64f4bad00653411821adcb75aea9015/speech\_command\_classification\_with\_torchaudio\_tutorial.ipynb\#scrollTo=i0pBRWkcxWrX}
\item \url{https://colab.research.google.com/github/tensorflow/docs/blob/master/site/en/tutorials/audio/simple\_audio.ipynb}
\end{enumerate}
\end{document}
